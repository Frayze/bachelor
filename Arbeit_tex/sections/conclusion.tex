\section{Fazit}
In dieser Arbeit wurden die grundlegenden Funktionen und Definitionen der Operatorpräzedenzsprachen zusammengefasst in Hinblick auf die Grammatiken und den Automaten. Weiterhin sind wichtige sprachtheoretische Eigenschaften beleuchtet und bewiesen worden, sowie in einer praktischen Anwendung implementiert worden. Allerdings gibt es in dem Bereich noch deutlich mehr Potenzial, was in dieser Bachelorarbeit nicht betrachtet wurde, aber zumindest an dieser Stelle als Ausblick erwähnt werden sollte: \begin{itemize}
\item
Eine monadische Prädikatenlogik zweiter Stufe  wurde für die OPL definiert \cite{mso}. 
\item
Die Sprache der OPL wurde erweitert zu $\omega$-OPL für unendliche Wörter. Auch dazu wurde die Prädikatenlogik zweiter Stufe entworfen.\cite{mso}.
\item
Ansätze für ModelChecking von Operatorpräzedenzsprachen \cite{modelchecking}
\item
Weitere Algebraische Eigenschaften der Operatorpräzedenzsprachen \cite{algebraic_properties}
\item
Betrachtung der Sprachen, mit Konflikt in der Matrix
\item
...
\end{itemize}
All das sind Gründe warum erneutes Interesse an diesen Sprachen angebracht wäre. 
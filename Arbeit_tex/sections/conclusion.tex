\section{Fazit}
Es war sehr spannend in ein, zwar nicht neues, aber durchaus noch unterrepräsentiertes Thema der Informatik einzutauchen. Ein erneutes Interesse an diesen Sprachen halte ich auf jeden Fall für angebracht. Trotzdem war es eine große Herausforderung in ein so theoretisches Themengebiet einzusteigen und die Definitionen und Beweise zu verinnerlichen. 
In dieser Arbeit wurden die grundlegenden Funktionen und Definitionen der Operatorpräzedenzsprachen zusammengefasst in Hinblick auf die Grammatiken und den Automaten. Weiterhin sind wichtige sprachtheoretische Eigenschaften beleuchtet und bewiesen, sowie in einer praktischen Anwendung implementiert worden. Allerdings gibt es in dem Bereich noch deutlich mehr Potenzial, was in dieser Bachelorarbeit nicht betrachtet wurde, aber zumindest an dieser Stelle als Ausblick erwähnt werden sollte: \begin{itemize}
\item
Eine monadische Prädikatenlogik zweiter Stufe  wurde für die OPL definiert \cite{mso}. 
\item
Die Sprache der OPL wurde erweitert zu $\omega$-OPL für unendliche Wörter. Auch dazu wurde die Prädikatenlogik zweiter Stufe entworfen.\cite{mso}.
\item
Ansätze für ModelChecking von Operatorpräzedenzsprachen \cite{modelchecking}
\item
Weitere Algebraische Eigenschaften der Operatorpräzedenzsprachen \cite{algebraic_properties}
\item
Betrachtung der Sprachen, mit Konflikt in der Matrix
\item
...
\end{itemize}
All das dürfte genügen, damit die Wissenschaft sich noch weiter mit den Operatorpäzedenzsprachen beschäftigen wird und man darf gespannt sein, was in den nächsten Jahren noch an weiteren Ergebnissen folgt. \\
Damit ist die Arbeit abgeschlossen.
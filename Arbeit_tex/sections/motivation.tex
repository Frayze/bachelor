\section{Motivation}
In dieser Arbeit geht es um die Operatorpräzedenzsprachen (OPL) und um die Eigenschaften ihrer Grammatiken und Automaten. Diese Sprachenklasse wurde 1963 von R.W.Floyd eingeführt, weswegen sie auch häufig als Floydsprachen oder Floydgrammatiken betitelt wird. Floyd interessierte die Struktur von Ausdrücken (sowhl einfache arithmetische als auch programmiertechnische) und die Präzedenz von manchen Operatoren über andere. Besonders interessant war dabei die Präzedenz der Multiplikation über die Addition, die per Konvention gilt und somit nicht explizit durch Klammern klargemacht werden muss. Er definiert dafür drei Präzedenzrelationen ($\lessdot, \gtrdot, \doteq$), die zwischen den Terminalsymbolen gelten und in einer Operatorpräzedenzmatrix (OPM) festgehalten werden.\\s
Als motivierendes Beispiel, was auch in den Definitionen weiterhin genutzt wird dient ein einfacher arithmetischer Ausdruck mit $\Sigma = \{n, +, \times, (, )\}$:\\ $ n \times (n + n \times n)$, bei dem die implizite Präzedenz deutlich wird.\\
Der Hauptaspekt hier liegt allerdings weniger auf den Grammatiken, als auf den Operatorpräzedenzautomaten (OPA), die deutlich neuer sind. Diese Automaten erkennen exakt die Sprachen, die von den Grammatiken generiert werden und umgekehrt. Die Äquivalenz wird in beide Richtungen gezeigt.
Generell bieten die OPLs viele interessante Eigenschaften: Sie sind eine echte Teilmenge der kontextfreien Sprachen, genießen aber trotzdem alle typischen Abschlusseigenschaften der regulären Sprache in Bezug auf Boolische Operatoren, Konkatenation und Kleene-*. 
Weiterhin kann gezeigt werden, dass die Klasse der Visibly-Pushdown Sprachen und Automaten ebenfalls von OPGs und OPAs erkannt werden. Im Gegensatz zu anderen Parsern wie zB. dem LR(1)- Parser muss hier nicht zwangsläufig von links nach rechts gearbeitet werden, was eine Möglichkeit zur Parallelisierung bietet. Ferner bieten OPLs auch eine Erweiterung zur $\omega- OPL$ und eine \textit{Monadic Second Order Logic Characterization}, was aber den Rahmen dieser Arbeit sprengen würde.
Als praktischer letzter Teil folgt dann eine Implementierung der OPAs in geeigneter Form.
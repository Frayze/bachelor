%!TEX TS-program = pdflatex
\documentclass[
%draft,
10pt,
%pantone312, 	% WWU-Design in hellblau
%pantone396, 	% WWU-Design in grellem hellgrün 
pantone315, 	% WWU-Design in dunkelblau
%pantone3282, 	% WWU-Design in "blau"
%pantone390, 	% WWU-Design in 
%pantone369,	% WWU-Design in gras-grün
%pantoneblack7, % WWU-Design in dunklem gras-grün
%blackberry,
%handout		% aktivieren um \pause für Druck zu ignorieren (keine neue Seite für einen neuen Punkt)
]{beamer}

\usepackage{wwustyle2}
\usepackage[ngerman]{babel}
\usepackage[utf8]{inputenc}
\usepackage[T1]{fontenc}%
\usepackage{multirow}
\usepackage{amsmath}
\usepackage{tabularx}
\usepackage{listings}
\usepackage{textcomp}
\usepackage{tikz-cd}
\usepackage{tikz}
\usetikzlibrary{quotes,babel,arrows,automata,positioning,trees,graphs,shapes,calc,decorations.pathreplacing}
\usepackage{multicol}
\usepackage[]{todonotes}
\usepackage{pgfpages}
\usepackage[%
backend=biber,
sortlocale=auto,
natbib,
hyperref,
backref,
style=alphabetic % eine unvollständige Auswahl von Styles: ieee, numeric, apa
]%
{biblatex}
\addbibresource{quellen.bib} % Literaturdatei einlesen
\setbeamertemplate{bibliography item}[text]


\newcommand{\bet}[1]{\textbf{\textcolor{maincolor}{#1}}}

\lstset{frame=tb,
	language=Java,
	aboveskip=3mm,
	belowskip=3mm,
	showstringspaces=false,
	columns=flexible,
	basicstyle={\small\ttfamily},
	numbers=left,
	frame=single,
	numberstyle=\tiny\color{gray},
	keywordstyle=\bfseries\color{maincolor!60!black},
	commentstyle=\itshape\color{orange!75!black},
	identifierstyle=\color{maincolor!20!black},
	stringstyle=\color{orange!98!black},
	backgroundcolor=\color{gray!10},	
	breaklines=true,
	breakatwhitespace=true,
	tabsize=3
}
\lstset{literate=%
	{Ö}{{\"O}}1
	{Ä}{{\"A}}1
	{Ü}{{\"U}}1
	{ß}{{\ss}}2
	{ü}{{\"u}}1
	{ä}{{\"a}}1
	{ö}{{\"o}}1
}

\newcommand{\java}[1]{\colorbox{gray!20}{\lstinline!#1!}}

\begin{document}
\date{\today}
\author{Jonas Kremer}
\title{Operator Precedence Languages}
\subtitle{Subtitle}

\setbeamertemplate{section in toc}[sections numbered]
\setbeamertemplate{subsection in toc}[subsections numbered]

%Wenn man vor jeder Section die Agenda sehen möchte...
%\AtBeginSection[]
%{
%	\begin{frame}[t]
%\tableofcontents[currentsection, hidesubsections, hideothersubsections,
%	sectionstyle=show/shaded]
%	\end{frame}
%}

\begin{frame}[plain]
  \maketitle
\end{frame}

\begin{frame}[t]{Agenda}
	\tableofcontents[hidesubsections, hideothersubsections]
\end{frame}

\section{Introduction}
\begin{frame}[t]{\secname}
	\begin{enumerate}[<+->]
		\item
	\end{enumerate}	
\end{frame}

\section{Preliminaries}
\begin{frame}[t]{Definition kontextfreie Grammatik}
	\begin{itemize}[<+->]
		\item
		Eine kontextfreie Grammatik ist ein 4-Tupel $G = (N, \Sigma , P, S)$
		\begin{itemize}[<+->]
			\item
			N ist die Menge der Nichtterminale
			\item
			$\Sigma$ ist die Menge der Terminalsymbole
			\item
			P sind die Produktionsregeln
			\item
			S $\in$ N ist das Startsymbol
		\end{itemize}
	\end{itemize}
\end{frame}
\begin{frame}[t]{Namenskonventionen 1}
	\begin{enumerate}
		\item
		$a,b,\dots\in\Sigma$ sind einzelne Terminalsymbole
		\item 
		$v,w,\dots\in\Sigma\textsuperscript{*}$ sind beliebige Terminalstrings
		\item
		$A,B,\dots\in N$ sind einzelne Nichtterminale
		\item
		$\alpha,\beta,\dots\in (\Sigma \cup N)\textsuperscript{*}$ sind beliebe Reststrings
		\item
		$A \rightarrow \epsilon $ ist die \textit{leere Regel}
		\item
		Eine \textit{umbenennende} Regel hat nur ein Nichtterminal als rechte Seite $\left( A \rightarrow B \right)$
	\end{enumerate}
\end{frame}

\begin{frame}[t]{Namenskonventionen 2}
	\begin{enumerate}
		\item
		Eine Regel ist in \textit{Operatorform}, wenn die rechte Seite keine benachbarten Nichtterminale hat
		\item 
		Jede kontextfreie Grammatik kann in eine äquivalente \textit{Operatorgrammatik (OG)} umgewandelt werden
		
	\end{enumerate}
\end{frame}

\section{Operator Precedence Languages}
\subsection{Operator Precedence Grammar}
\begin{frame}[t]{\subsecname}
	\begin{itemize}[<+->]
		\item
		Wird häufig auch \textit{Floyd Grammar} genannt nach ihrem Erfinder
		\item
		Definition: \textit{Linke und Rechte Terminalmenge}\\
		$\mathcal{L} \textsubscript{G} \big( A \big) = \{ a \in \Sigma | 
		A \overset{*}{\Rightarrow}Ba \alpha \} $\\
		$\mathcal{R} \textsubscript{G} \big( A \big) = \{ a \in \Sigma | 
		A \overset{*}{\Rightarrow} \alpha aB \}$
		\item
		Es werden drei binäre Operator Precedence Relationen definiert:
		\item
		equal in precedence: $ a \doteq b \Leftrightarrow \exists A \rightarrow \alpha aBb \beta , 
		B \in N \cup \{ \epsilon \}$ \\
		takes precedence: $ a \gtrdot b \Leftrightarrow \exists A \rightarrow \alpha Db \beta , D \in N $ and $ a \in
		\mathcal{R}\textsubscript{G}(D)$ \\
		yields precedence: $ a \lessdot b \Leftrightarrow \exists A \rightarrow \alpha aD \beta , D \in N $ and $ b \in
		\mathcal{L}\textsubscript{G}(D)$
	\end{itemize}
\end{frame}

\subsection{Operator Precedence Automata}
\begin{frame}[t]{\secname}
	\begin{itemize}[<+->]
		\item
		Ein nichtdetermistischer Operator Precedence Automat ist ein 6-Tupel A = $(\Sigma, M, Q, I, F, \delta)$
		\begin{enumerate}
			\item
			$(\Sigma, M)$ ist ein OP alphabet
			\item
			Q ist die Menge der Zustände
			\item
			$ I \subseteq Q$ ist die Menge der Startzustände
			\item
			$ F \subseteq Q$ ist die Menge der finalen Zustände
			\item 
			$\delta$ ist die Übergangsfunktion, die aus drei Teilen besteht: \\
			$\delta \textsubscript{shift}: Q \times \Sigma \rightarrow \mathcal{P} (Q)$
			$\delta \textsubscript{push}: Q \times \Sigma \rightarrow \mathcal{P} (Q)$
			$\delta \textsubscript{pop}: Q \times Q \rightarrow \mathcal{P} (Q)$
		\end{enumerate}
			
	\end{itemize}
\end{frame}

\section{(Closure-)Properties}
\begin{frame}[t]{\secname}
	\begin{itemize}[<+->]
	\item 
	OPLs sind eine große Subklasse der kontextfreien Sprachen, die Abgeschlossenheitseigenschaften von regulären Sprachen genießt
	\item
	Abgeschlossen unter Vereinigung, Schnitt, Komplement, Konkatenation und Kleene-*
	\item
	Das Leereproblem ist in PTIME lösbar, da OPLs Subklasse von kfG
	\item Visibly Pushdown Sprachen sind in der Klasse der OPLs enthalten
	\end{itemize}
\end{frame}

\nocite{*}
\section*{Quellen}
\begin{frame}{Quellen}
	\printbibliography
\end{frame}

\end{document}
